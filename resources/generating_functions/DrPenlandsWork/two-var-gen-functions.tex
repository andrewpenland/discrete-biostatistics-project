\documentclass[11pt]{article}

\usepackage{amsmath, amssymb}

\newcommand{\tallbinom}[2]{\left( \begin{array}{c} {#1} \\[8pt] {#2} \end{array} \right)}

\begin{document}

\textbf{Some Notes on Two-Variable Generating Functions}

At this point, we have gone from recurrence relations to generating functions. Now, our goal is to take generating functions and simplify them to obtain a form of final answer. 

\textbf{Notation:} 
\begin{itemize}
\item For me, $\mathbb{N}$ includes $0$.
\item We define the \textit{protected binomial coefficient} $\binom{n}{k}_p$ to be equal to $\binom{n}{k}$ if both $n, k \in \mathbb{N}$, and $0$ otherwise. 
\end{itemize}

\hrule

\textbf{Example 1:} Our goal is to look at the generating function
\[
D(x,y) = \displaystyle\frac{1}{1 - x^2y - xy^3}.
\]
and find the coefficient on $x^n y^j$ for a specific value of $n$ and $j$. 


\textbf{Solution:} Factoring and using binomial theorem, we have

\begin{align*}
D(x,y) &= \displaystyle\frac{1}{1 - x^2y - xy^3} \\
&= \displaystyle\frac{1}{1 - xy(x + y^2)} \\
&= \displaystyle\sum_{m=0}^{\infty} \left[xy(x+y^2) \right]^m \\
&= \displaystyle\sum_{m=0}^\infty (xy)^m (x+y^2)^m. \\
\end{align*}

Now we apply the binomial theorem on $(x+y^2)^m$, use the distributive property, and simplify the exponents. 

\begin{align*}
D(x,y) &= \displaystyle\sum_{m=0}^\infty (xy)^m \left( \binom{m}{r} x^r (y^2)^{m-r} \right) \\ 
&= \displaystyle\sum_{m=0}^\infty (xy)^m \left[ \displaystyle\sum_{r=0}^m \binom{m}{r} x^r y^{2m - 2r}  \right] \\
&= \displaystyle\sum_{m=0}^\infty \displaystyle\sum_{r = 0}^m \binom{m}{r} x^{m+r} y^{3m - 2r} \\ 
\end{align*}

Now we fix specific values of $n$ and $j$ to find the coefficient on $x^n y^j$. 

To extract this coefficient, we assume

\[
n = m + r \text{ and } j = 3m - 2r.
\]

Here is the \textbf{KEY INSIGHT}: we can view this as a system of equations over the real numbers. From linear algebra, this system of equations has at most one $(m,r) \in \mathbb{R}^2$ that satisfies it. With the further restrictions on $m$ and $r$ ($m, r \in \mathbb{N}$, $0 \leq r \leq m$), it is possible that the system has no solution. 

If this system does have a solution, it would be given by 
\begin{align*}
m &= n - r \\
r &= \displaystyle\frac{3n - j}{5} 
\end{align*}

The coefficient on $x^{m+r} y^{3m - 2r}$ is $\binom{m}{r}$, so the coefficient on $x^n y^j$ is 

\[
\tallbinom{n - \frac{3n - j}{5}}{\frac{3n - j}{5}} \text{ if } \frac{3n - j}{5} \in\mathbb{N}
\]

and 0 otherwise. 

Rewriting this with a common denominator and our protected binomial notation, we can say that the coefficient on $x^n y^j$ is

\[
\tallbinom{\frac{2n+j}{5}}{\frac{3n-j}{5}}_p.
\]

\hrule

\textbf{Example 2:} We want to find the coefficient on $x^n y^j$ ($n, j \in \mathbb{N}$) for 
\[
G(x,y) = \displaystyle\frac{1}{1 - x - x^2y - xy^3}. 
\]

Following a similar process as Example 1 (except we use binomial theorem twice), we have this calculation: \\

\begin{align*}
G(x,y) &= \displaystyle\frac{1}{1 - x(1 + xy + y^2)} \\
&= \displaystyle\sum_{m = 0}^\infty \left( x (1+ xy + y^2)) \right)^m \\
&= \displaystyle\sum_{m = 0}^\infty x^m \left(1 + xy + y^2 \right)^m \\
&= \displaystyle\sum_{m = 0}^\infty x^m  \left( \displaystyle\sum_{r=0}^m \binom{m}{r} 1^{m-r} (xy + y^2)^r \right) \\
&= \displaystyle\sum_{m = 0}^\infty x^m \left( \displaystyle\sum_{r = 0}^m \binom{m}{r} \displaystyle\sum_{t = 0}^r \binom{r}{t} (xy)^t (y^2)^{r-t}  \right) \\
&= \displaystyle\sum_{m = 0}^\infty \displaystyle\sum_{r = 0}^m \displaystyle\sum_{t = 0}^r \binom{m}{r} \binom{r}{t} x^{m+t} y^{2r - t} \\
\end{align*}

Now we fix specific non-negative integers $n$ and $j$, hoping to find the coefficient on $x^n y^j$. 

If $n = m + t$, $j = 2r - t$, this produces a system of equations over the real numbers which is \textit{underdetermined}: $t$ will be a free variable. However, we are restricted to the case where $t \in \{0, 1, 2, \ldots, n\}$. 

In terms of the free variable $t$, we have $m = n - t$ and $r = \displaystyle\frac{j + t}{2}$. So the coefficient on $x^n y^j$ will be given by 
\[
\tallbinom{n -t}{\frac{j+t}{2}} \tallbinom{ \frac{j+t}{2}}{t} 
\]

as long as the quantities involved in the binomial coefficient are integers. We rewrite this in terms of the protected binomial coefficient to obtain our final answer. 

\hrule 

\textbf{Example 3:} We want to find the coefficient on $x^n y^j$ ($n, j \in \mathbb{N}$) for 
\[
G(x,y) = \displaystyle\frac{1 + 3x}{1 - x - x^2y - xy^3}. 
\]

Notice that this has the same denominator as Example 2, but a different numerator. We will not repeat all of the algebra, but we notice that distributing the $(1 + 3x)$ through the terms obtained by expanding the denominator as a power series produces the following sum: \\

\[
 \displaystyle\sum_{m = 0}^\infty \displaystyle\sum_{r = 0}^m \displaystyle\sum_{t = 0}^r \binom{m}{r} \binom{r}{t} x^{m+t} y^{2r - t} + 3x^{m+t+1}y^{2r - t}.
\]

Now, in solving for the coefficient on $x^n y^j$, we have two possibilities, $n = m + t$ or $n = m + t + 1$. Of course, the two cases are similar, but for a fixed value of $t$ we will end up with a sum of two terms (each of which is the product of two binomial coefficients). 

I obtained 

\[
\displaystyle\sum_{t = 0}^n \tallbinom{n - t}{\frac{j + t}{2}}_p \tallbinom{\frac{j + t}{2}}{t}_p + \tallbinom{n - t - 1}{\frac{j + t}{2}}_p \tallbinom{\frac{j + t}{2}}{t}_p
\]

as my final answer. 

\textbf{Comment:} If the denominator contains even more terms, we will likely pay the price with more summations, and more free variables. This is to be expected.

An alternative for expressing the protected binomial coefficients would be to use multinomial coefficients instead of binomial coefficients. 

\end{document} 

